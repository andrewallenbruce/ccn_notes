% Options for packages loaded elsewhere
% Options for packages loaded elsewhere
\PassOptionsToPackage{unicode}{hyperref}
\PassOptionsToPackage{hyphens}{url}
\PassOptionsToPackage{dvipsnames,svgnames,x11names}{xcolor}
%
\documentclass[
  letterpaper,
  DIV=11,
  numbers=noendperiod]{scrreprt}
\usepackage{xcolor}
\usepackage{amsmath,amssymb}
\setcounter{secnumdepth}{5}
\usepackage{iftex}
\ifPDFTeX
  \usepackage[T1]{fontenc}
  \usepackage[utf8]{inputenc}
  \usepackage{textcomp} % provide euro and other symbols
\else % if luatex or xetex
  \usepackage{unicode-math} % this also loads fontspec
  \defaultfontfeatures{Scale=MatchLowercase}
  \defaultfontfeatures[\rmfamily]{Ligatures=TeX,Scale=1}
\fi
\usepackage{lmodern}
\ifPDFTeX\else
  % xetex/luatex font selection
\fi
% Use upquote if available, for straight quotes in verbatim environments
\IfFileExists{upquote.sty}{\usepackage{upquote}}{}
\IfFileExists{microtype.sty}{% use microtype if available
  \usepackage[]{microtype}
  \UseMicrotypeSet[protrusion]{basicmath} % disable protrusion for tt fonts
}{}
\makeatletter
\@ifundefined{KOMAClassName}{% if non-KOMA class
  \IfFileExists{parskip.sty}{%
    \usepackage{parskip}
  }{% else
    \setlength{\parindent}{0pt}
    \setlength{\parskip}{6pt plus 2pt minus 1pt}}
}{% if KOMA class
  \KOMAoptions{parskip=half}}
\makeatother
% Make \paragraph and \subparagraph free-standing
\makeatletter
\ifx\paragraph\undefined\else
  \let\oldparagraph\paragraph
  \renewcommand{\paragraph}{
    \@ifstar
      \xxxParagraphStar
      \xxxParagraphNoStar
  }
  \newcommand{\xxxParagraphStar}[1]{\oldparagraph*{#1}\mbox{}}
  \newcommand{\xxxParagraphNoStar}[1]{\oldparagraph{#1}\mbox{}}
\fi
\ifx\subparagraph\undefined\else
  \let\oldsubparagraph\subparagraph
  \renewcommand{\subparagraph}{
    \@ifstar
      \xxxSubParagraphStar
      \xxxSubParagraphNoStar
  }
  \newcommand{\xxxSubParagraphStar}[1]{\oldsubparagraph*{#1}\mbox{}}
  \newcommand{\xxxSubParagraphNoStar}[1]{\oldsubparagraph{#1}\mbox{}}
\fi
\makeatother

\usepackage{color}
\usepackage{fancyvrb}
\newcommand{\VerbBar}{|}
\newcommand{\VERB}{\Verb[commandchars=\\\{\}]}
\DefineVerbatimEnvironment{Highlighting}{Verbatim}{commandchars=\\\{\}}
% Add ',fontsize=\small' for more characters per line
\usepackage{framed}
\definecolor{shadecolor}{RGB}{241,243,245}
\newenvironment{Shaded}{\begin{snugshade}}{\end{snugshade}}
\newcommand{\AlertTok}[1]{\textcolor[rgb]{0.68,0.00,0.00}{#1}}
\newcommand{\AnnotationTok}[1]{\textcolor[rgb]{0.37,0.37,0.37}{#1}}
\newcommand{\AttributeTok}[1]{\textcolor[rgb]{0.40,0.45,0.13}{#1}}
\newcommand{\BaseNTok}[1]{\textcolor[rgb]{0.68,0.00,0.00}{#1}}
\newcommand{\BuiltInTok}[1]{\textcolor[rgb]{0.00,0.23,0.31}{#1}}
\newcommand{\CharTok}[1]{\textcolor[rgb]{0.13,0.47,0.30}{#1}}
\newcommand{\CommentTok}[1]{\textcolor[rgb]{0.37,0.37,0.37}{#1}}
\newcommand{\CommentVarTok}[1]{\textcolor[rgb]{0.37,0.37,0.37}{\textit{#1}}}
\newcommand{\ConstantTok}[1]{\textcolor[rgb]{0.56,0.35,0.01}{#1}}
\newcommand{\ControlFlowTok}[1]{\textcolor[rgb]{0.00,0.23,0.31}{\textbf{#1}}}
\newcommand{\DataTypeTok}[1]{\textcolor[rgb]{0.68,0.00,0.00}{#1}}
\newcommand{\DecValTok}[1]{\textcolor[rgb]{0.68,0.00,0.00}{#1}}
\newcommand{\DocumentationTok}[1]{\textcolor[rgb]{0.37,0.37,0.37}{\textit{#1}}}
\newcommand{\ErrorTok}[1]{\textcolor[rgb]{0.68,0.00,0.00}{#1}}
\newcommand{\ExtensionTok}[1]{\textcolor[rgb]{0.00,0.23,0.31}{#1}}
\newcommand{\FloatTok}[1]{\textcolor[rgb]{0.68,0.00,0.00}{#1}}
\newcommand{\FunctionTok}[1]{\textcolor[rgb]{0.28,0.35,0.67}{#1}}
\newcommand{\ImportTok}[1]{\textcolor[rgb]{0.00,0.46,0.62}{#1}}
\newcommand{\InformationTok}[1]{\textcolor[rgb]{0.37,0.37,0.37}{#1}}
\newcommand{\KeywordTok}[1]{\textcolor[rgb]{0.00,0.23,0.31}{\textbf{#1}}}
\newcommand{\NormalTok}[1]{\textcolor[rgb]{0.00,0.23,0.31}{#1}}
\newcommand{\OperatorTok}[1]{\textcolor[rgb]{0.37,0.37,0.37}{#1}}
\newcommand{\OtherTok}[1]{\textcolor[rgb]{0.00,0.23,0.31}{#1}}
\newcommand{\PreprocessorTok}[1]{\textcolor[rgb]{0.68,0.00,0.00}{#1}}
\newcommand{\RegionMarkerTok}[1]{\textcolor[rgb]{0.00,0.23,0.31}{#1}}
\newcommand{\SpecialCharTok}[1]{\textcolor[rgb]{0.37,0.37,0.37}{#1}}
\newcommand{\SpecialStringTok}[1]{\textcolor[rgb]{0.13,0.47,0.30}{#1}}
\newcommand{\StringTok}[1]{\textcolor[rgb]{0.13,0.47,0.30}{#1}}
\newcommand{\VariableTok}[1]{\textcolor[rgb]{0.07,0.07,0.07}{#1}}
\newcommand{\VerbatimStringTok}[1]{\textcolor[rgb]{0.13,0.47,0.30}{#1}}
\newcommand{\WarningTok}[1]{\textcolor[rgb]{0.37,0.37,0.37}{\textit{#1}}}

\usepackage{longtable,booktabs,array}
\usepackage{calc} % for calculating minipage widths
% Correct order of tables after \paragraph or \subparagraph
\usepackage{etoolbox}
\makeatletter
\patchcmd\longtable{\par}{\if@noskipsec\mbox{}\fi\par}{}{}
\makeatother
% Allow footnotes in longtable head/foot
\IfFileExists{footnotehyper.sty}{\usepackage{footnotehyper}}{\usepackage{footnote}}
\makesavenoteenv{longtable}
\usepackage{graphicx}
\makeatletter
\newsavebox\pandoc@box
\newcommand*\pandocbounded[1]{% scales image to fit in text height/width
  \sbox\pandoc@box{#1}%
  \Gscale@div\@tempa{\textheight}{\dimexpr\ht\pandoc@box+\dp\pandoc@box\relax}%
  \Gscale@div\@tempb{\linewidth}{\wd\pandoc@box}%
  \ifdim\@tempb\p@<\@tempa\p@\let\@tempa\@tempb\fi% select the smaller of both
  \ifdim\@tempa\p@<\p@\scalebox{\@tempa}{\usebox\pandoc@box}%
  \else\usebox{\pandoc@box}%
  \fi%
}
% Set default figure placement to htbp
\def\fps@figure{htbp}
\makeatother





\setlength{\emergencystretch}{3em} % prevent overfull lines

\providecommand{\tightlist}{%
  \setlength{\itemsep}{0pt}\setlength{\parskip}{0pt}}



 


\usepackage{booktabs}
\usepackage{caption}
\usepackage{longtable}
\usepackage{colortbl}
\usepackage{array}
\usepackage{anyfontsize}
\usepackage{multirow}
\KOMAoption{captions}{tableheading}
\makeatletter
\@ifpackageloaded{tcolorbox}{}{\usepackage[skins,breakable]{tcolorbox}}
\@ifpackageloaded{fontawesome5}{}{\usepackage{fontawesome5}}
\definecolor{quarto-callout-color}{HTML}{909090}
\definecolor{quarto-callout-note-color}{HTML}{0758E5}
\definecolor{quarto-callout-important-color}{HTML}{CC1914}
\definecolor{quarto-callout-warning-color}{HTML}{EB9113}
\definecolor{quarto-callout-tip-color}{HTML}{00A047}
\definecolor{quarto-callout-caution-color}{HTML}{FC5300}
\definecolor{quarto-callout-color-frame}{HTML}{acacac}
\definecolor{quarto-callout-note-color-frame}{HTML}{4582ec}
\definecolor{quarto-callout-important-color-frame}{HTML}{d9534f}
\definecolor{quarto-callout-warning-color-frame}{HTML}{f0ad4e}
\definecolor{quarto-callout-tip-color-frame}{HTML}{02b875}
\definecolor{quarto-callout-caution-color-frame}{HTML}{fd7e14}
\makeatother
\makeatletter
\@ifpackageloaded{bookmark}{}{\usepackage{bookmark}}
\makeatother
\makeatletter
\@ifpackageloaded{caption}{}{\usepackage{caption}}
\AtBeginDocument{%
\ifdefined\contentsname
  \renewcommand*\contentsname{Table of contents}
\else
  \newcommand\contentsname{Table of contents}
\fi
\ifdefined\listfigurename
  \renewcommand*\listfigurename{List of Figures}
\else
  \newcommand\listfigurename{List of Figures}
\fi
\ifdefined\listtablename
  \renewcommand*\listtablename{List of Tables}
\else
  \newcommand\listtablename{List of Tables}
\fi
\ifdefined\figurename
  \renewcommand*\figurename{Figure}
\else
  \newcommand\figurename{Figure}
\fi
\ifdefined\tablename
  \renewcommand*\tablename{Table}
\else
  \newcommand\tablename{Table}
\fi
}
\@ifpackageloaded{float}{}{\usepackage{float}}
\floatstyle{ruled}
\@ifundefined{c@chapter}{\newfloat{codelisting}{h}{lop}}{\newfloat{codelisting}{h}{lop}[chapter]}
\floatname{codelisting}{Listing}
\newcommand*\listoflistings{\listof{codelisting}{List of Listings}}
\makeatother
\makeatletter
\makeatother
\makeatletter
\@ifpackageloaded{caption}{}{\usepackage{caption}}
\@ifpackageloaded{subcaption}{}{\usepackage{subcaption}}
\makeatother
\usepackage{bookmark}
\IfFileExists{xurl.sty}{\usepackage{xurl}}{} % add URL line breaks if available
\urlstyle{same}
\hypersetup{
  pdftitle={CMS Certification Numbers},
  pdfauthor={Andrew Allen Bruce},
  colorlinks=true,
  linkcolor={blue},
  filecolor={Maroon},
  citecolor={Blue},
  urlcolor={Blue},
  pdfcreator={LaTeX via pandoc}}


\title{CMS Certification Numbers}
\author{Andrew Allen Bruce}
\date{2025-12-24}
\begin{document}
\maketitle

\renewcommand*\contentsname{Table of contents}
{
\hypersetup{linkcolor=}
\setcounter{tocdepth}{2}
\tableofcontents
}

\bookmarksetup{startatroot}

\chapter*{Preface}\label{preface}
\addcontentsline{toc}{chapter}{Preface}

\markboth{Preface}{Preface}

This is a Quarto book.

To learn more about Quarto books visit
\url{https://quarto.org/docs/books}.

\begin{Shaded}
\begin{Highlighting}[]
\DecValTok{1} \SpecialCharTok{+} \DecValTok{1}
\end{Highlighting}
\end{Shaded}

\begin{verbatim}
[1] 2
\end{verbatim}

\bookmarksetup{startatroot}

\chapter{Introduction}\label{introduction}

\begin{Shaded}
\begin{Highlighting}[]
\FunctionTok{library}\NormalTok{(ccn)}
\end{Highlighting}
\end{Shaded}

A \textbf{CMS Certification Number} is a uniform way of identifying
Healthcare Providers or Suppliers who have participated in the Medicare
or Medicaid programs.

The CCN is used to identify each separately certified Medicare provider
or supplier. It is used to track provider agreements and cost reports.
The national provider identifier (NPI) and provider transaction account
number (PTAN) are tied to the CCN.

Additionally, CMS data systems use the CCN to identify each individual
provider or supplier that has or currently does participate in Medicare
and/or Medicaid. The RO, not the SA or MAC, assigns the CCN and
maintains adequate controls.

\begin{tcolorbox}[enhanced jigsaw, toprule=.15mm, left=2mm, colframe=quarto-callout-note-color-frame, breakable, leftrule=.75mm, opacitybacktitle=0.6, bottomtitle=1mm, toptitle=1mm, rightrule=.15mm, titlerule=0mm, title=\textcolor{quarto-callout-note-color}{\faInfo}\hspace{0.5em}{CCN}, colbacktitle=quarto-callout-note-color!10!white, arc=.35mm, bottomrule=.15mm, colback=white, opacityback=0, coltitle=black]

\textbf{CCN} replaced the terms \emph{Medicare Provider Number},
\emph{Medicare Identification Number} and \emph{OSCAR Number} in
\href{https://www.cms.gov/medicare/provider-enrollment-and-certification/surveycertificationgeninfo/policy-and-memos-to-states-and-regions-items/cms1206256}{2007}.

\end{tcolorbox}

\section{Standard Format}\label{standard-format}

A canonically valid CCN is an alphanumeric string, either 6 or 10
characters in length, that encodes a provider/supplier's state and
facility type.

\section{States}\label{states}

The first two characters of the CCN represent the state in which the
provider or supplier is located.

\begin{verbatim}
Registered S3 method overwritten by 'grid':
  method     from
  print.unit ccn 
\end{verbatim}

\begin{table}
\fontsize{12.0pt}{14.0pt}\selectfont
\begin{tabular*}{\linewidth}{@{\extracolsep{\fill}}lll}
\toprule
{\fontsize{10}{13}\selectfont ABBR} & {\fontsize{10}{13}\selectfont NAME} & {\fontsize{10}{13}\selectfont CODE} \\ 
\midrule\addlinespace[2.5pt]
AK & Alaska & 02 \\ 
AL & Alabama & 01 \\ 
AR & Arkansas & 04, 89 \\ 
AS & American Samoa & 64 \\ 
AZ & Arizona & 00, 03 \\ 
CA & California & 05, 55, 75, 92, A0, A1, B2 \\ 
CN & Canada & 56 \\ 
CO & Colorado & 06, 91 \\ 
CT & Connecticut & 07, 81 \\ 
DC & District of Columbia & 09 \\ 
DE & Delaware & 08 \\ 
FC & Foreign Country & 99 \\ 
FL & Florida & 10, 68, 69, A2 \\ 
GA & Georgia & 11, 85 \\ 
GU & Guam & 65 \\ 
HI & Hawaii & 12 \\ 
IA & Iowa & 16, 76 \\ 
ID & Idaho & 13, 54 \\ 
IL & Illinois & 14, 78 \\ 
IN & Indiana & 15 \\ 
KS & Kansas & 17, 70 \\ 
KY & Kentucky & 18, B0 \\ 
LA & Louisiana & 19, 71, 95, A3 \\ 
MA & Massachusetts & 22, 82 \\ 
MD & Maryland & 21, 80 \\ 
ME & Maine & 20 \\ 
MI & Michigan & 23, A4 \\ 
MN & Minnesota & 24, 77 \\ 
MO & Missouri & 26, 79 \\ 
MP & Northern Marianas Islands & 66 \\ 
MS & Mississippi & 25, A5 \\ 
MT & Montana & 27 \\ 
MX & Mexico & 59 \\ 
NC & North Carolina & 34, 86 \\ 
ND & North Dakota & 35 \\ 
NE & Nebraska & 28 \\ 
NH & New Hampshire & 30 \\ 
NJ & New Jersey & 31, 83 \\ 
NM & New Mexico & 32, 96 \\ 
NV & Nevada & 29 \\ 
NY & New York & 33, 57 \\ 
OH & Ohio & 36, 72, A6 \\ 
OK & Oklahoma & 37, 90 \\ 
OR & Oregon & 38, 93 \\ 
PA & Pennsylvania & 39, 73, A7 \\ 
PR & Puerto Rico & 40, 84 \\ 
RI & Rhode Island & 41 \\ 
SC & South Carolina & 42, 87 \\ 
SD & South Dakota & 43 \\ 
TN & Tennessee & 44, 88, A8 \\ 
TX & Texas & 45, 67, 74, 97, A9 \\ 
UT & Utah & 46 \\ 
VA & Virginia & 49 \\ 
VI & Virgin Islands & 48 \\ 
VT & Vermont & 47 \\ 
WA & Washington & 50, 94 \\ 
WI & Wisconsin & 52 \\ 
WV & West Virginia & 51, 58, B1 \\ 
WY & Wyoming & 53 \\ 
\bottomrule
\end{tabular*}
\end{table}

\subsection{Regional Offices}\label{regional-offices}

\begin{table}
\fontsize{12.0pt}{14.0pt}\selectfont
\begin{tabular*}{\linewidth}{@{\extracolsep{\fill}}lll}
\toprule
{\fontsize{10}{13}\selectfont REGION} & {\fontsize{10}{13}\selectfont OFFICE} & {\fontsize{10}{13}\selectfont STATE} \\ 
\midrule\addlinespace[2.5pt]
I & Boston & CT, MA, ME, NH, RI, VT \\ 
II & New York & NJ, NY, PR, VI \\ 
III & Philadelphia & DC, DE, MD, PA, VA, WV \\ 
IV & Atlanta & AL, FL, GA, KY, MS, NC, SC, TN \\ 
IX & San Francisco & AS, AZ, CA, GU, HI, MP, NV \\ 
V & Chicago & IL, IN, MI, MN, OH, WI \\ 
VI & Dallas & AR, LA, NM, OK, TX \\ 
VII & Kansas City & IA, KS, MO, NE \\ 
VIII & Denver & CO, MT, ND, SD, UT, WY \\ 
X & Seattle & AK, ID, OR, WA \\ 
\bottomrule
\end{tabular*}
\end{table}

\bookmarksetup{startatroot}

\chapter{Medicare}\label{medicare}

\begin{Shaded}
\begin{Highlighting}[]
\FunctionTok{library}\NormalTok{(ccn)}
\end{Highlighting}
\end{Shaded}

\section{Ranges}\label{ranges}

The first two characters of the CCN represent the state in which the
provider or supplier is located.

\begin{verbatim}
Registered S3 method overwritten by 'grid':
  method     from
  print.unit ccn 
\end{verbatim}

\begin{table}
\fontsize{12.0pt}{14.0pt}\selectfont
\begin{tabular*}{\linewidth}{@{\extracolsep{\fill}}llr}
\toprule
{\fontsize{10}{13}\selectfont ABBR} & {\fontsize{10}{13}\selectfont DESC} & {\fontsize{10}{13}\selectfont RANGE} \\ 
\midrule\addlinespace[2.5pt]
ADH (Ret.) & Alcohol-Drug Hospital (Retired) & 1200-1224 \\ 
CAH & Critical Access Hospital & 1300-1399 \\ 
CH & Children's Hospital & 3300-3399 \\ 
CMHC & Community Mental Health Center & 1400-1499, 4600-4799, 4900-4999 \\ 
CORF & Comprehensive Outpatient Rehabilitation Facility & 3200-3299, 4500-4599, 4800-4899 \\ 
FQHC & Federally Qualified Health Center & 1000-1199, 1800-1989 \\ 
HBRDF & Hospital-based Renal Dialysis Facility & 2300-2499 \\ 
HBSRDF & Hospital-based Satellite Renal Dialysis Facility & 3500-3699 \\ 
HHA & Home Health Agency & 7000-7299, 7400-7799, 8000-8499, 9000-9799 \\ 
HHA & Home Health Agency Subunit (Nonprofit/Proprietary) & 3100-3199, 7300-7399 \\ 
HHA & Home Health Agency Subunit (State/Local) & 7800-7999 \\ 
HSP & Hospice & 1500-1799 \\ 
HSPRDF & Hospital-based Special Purpose Renal Dialysis Facility & 3700-3799 \\ 
IRDF & Independent Renal Dialysis Facility & 2500-2899 \\ 
IRF & Rehabilitation Hospital & 3025-3099 \\ 
ISPRDF & Independent Special Purpose Renal Dialysis Facility & 2900-2999 \\ 
LTCH & Long-Term Care Hospital & 2000-2299 \\ 
MAF & Medical Assistance Facility & 1225-1299 \\ 
MHCMC (Ret.) & Multiple Hospital Component in a Medical Complex (Retired) & 0900-0999 \\ 
OCM & Hospital Participating in ORD (Oncology Care Model) Demonstration Project & 0880-0899 \\ 
OPT & Outpatient Physical Therapy Services & 6500-6989 \\ 
OTP & Freestanding Opioid Treatment Program & 9900-9999 \\ 
PH & Psychiatric Hospital & 4000-4499 \\ 
RHC & Rural Health Clinic (Free-standing) & 3800-3974, 8900-8999 \\ 
RHC & Rural Health Clinic (Provider-based) & 3400-3499, 3975-3999, 8500-8899 \\ 
RNHCI & Religious Non-medical Health Care Institution & 1990-1999 \\ 
RSVD & Number Reserved (Formerly CSS) & 6990-6999 \\ 
SNF & Skilled Nursing Facility & 5000-6499 \\ 
STC & Short-Term Hospital (General \& Specialty) & 0001-0879 \\ 
TBH (Ret.) & Tuberculosis Hospital (Retired) & 3000-3024 \\ 
TXC & Transplant Center & 9800-9899 \\ 
\bottomrule
\end{tabular*}
\end{table}

\bookmarksetup{startatroot}

\chapter*{References}\label{references}
\addcontentsline{toc}{chapter}{References}

\markboth{References}{References}

\phantomsection\label{refs}




\end{document}
